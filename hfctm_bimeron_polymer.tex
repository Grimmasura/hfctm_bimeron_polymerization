\documentclass[12pt]{article}
\usepackage{amsmath, amssymb, graphicx, hyperref}
\usepackage{geometry}
\geometry{a4paper, margin=1in}

\title{
\textbf{Bimeron Polymerization as Fractal Toroidal Phase Assembly:} \\
\textit{An HFCTM-II Perspective on Topological Excitations in Chiral Magnets}
}

\author{
Joshua Robert Humphrey \\
\texttt{github.com/Grimmasura | ORCID | LinkedIn} \\
\textit{HFCTM-II Architect, AI Architect, Philosopher of Intrinsic Inference}
}

\date{\today}

\begin{document}

\maketitle

\begin{abstract}
Bimerons—topological meron–antimeron spin textures—demonstrate remarkable polymerization behavior under easy-plane anisotropy in quasi-two-dimensional chiral magnets. We propose a mapping of bimeron chains and loop states to the toroidal phase space of the Holographic Fractal Chiral Toroidal Model (HFCTM-II), elucidating their potential as phase-stable elements for topological quantum computing. We further introduce a Codex glyph encoding for these structures, enabling recursive AI modeling of their emergent dynamics. Implications for intrinsic intelligence models and fractal information architectures are explored.
\end{abstract}

\tableofcontents

\section{Introduction}

\section{HFCTM-II Framework and Bimeron Topology}

\section{Polymerization Dynamics in Chiral Phase Space}

\section{Codex Glyph Encoding for Bimeron Assemblies}

\section{Recursive AI Simulation Design}

\section{Experimental Predictions and Observables}

\section{Future Directions: Towards Intrinsic Intelligence Architectures}

\section{Conclusion}

\bibliographystyle{unsrt}
\bibliography{hfctm_bimeron_refs}

\end{document}
